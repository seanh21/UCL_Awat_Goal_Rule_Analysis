% Options for packages loaded elsewhere
\PassOptionsToPackage{unicode}{hyperref}
\PassOptionsToPackage{hyphens}{url}
%
\documentclass[
]{article}
\usepackage{amsmath,amssymb}
\usepackage{lmodern}
\usepackage{ifxetex,ifluatex}
\ifnum 0\ifxetex 1\fi\ifluatex 1\fi=0 % if pdftex
  \usepackage[T1]{fontenc}
  \usepackage[utf8]{inputenc}
  \usepackage{textcomp} % provide euro and other symbols
\else % if luatex or xetex
  \usepackage{unicode-math}
  \defaultfontfeatures{Scale=MatchLowercase}
  \defaultfontfeatures[\rmfamily]{Ligatures=TeX,Scale=1}
\fi
% Use upquote if available, for straight quotes in verbatim environments
\IfFileExists{upquote.sty}{\usepackage{upquote}}{}
\IfFileExists{microtype.sty}{% use microtype if available
  \usepackage[]{microtype}
  \UseMicrotypeSet[protrusion]{basicmath} % disable protrusion for tt fonts
}{}
\makeatletter
\@ifundefined{KOMAClassName}{% if non-KOMA class
  \IfFileExists{parskip.sty}{%
    \usepackage{parskip}
  }{% else
    \setlength{\parindent}{0pt}
    \setlength{\parskip}{6pt plus 2pt minus 1pt}}
}{% if KOMA class
  \KOMAoptions{parskip=half}}
\makeatother
\usepackage{xcolor}
\IfFileExists{xurl.sty}{\usepackage{xurl}}{} % add URL line breaks if available
\IfFileExists{bookmark.sty}{\usepackage{bookmark}}{\usepackage{hyperref}}
\hypersetup{
  pdftitle={UCL policy project markdown},
  hidelinks,
  pdfcreator={LaTeX via pandoc}}
\urlstyle{same} % disable monospaced font for URLs
\usepackage[margin=1in]{geometry}
\usepackage{color}
\usepackage{fancyvrb}
\newcommand{\VerbBar}{|}
\newcommand{\VERB}{\Verb[commandchars=\\\{\}]}
\DefineVerbatimEnvironment{Highlighting}{Verbatim}{commandchars=\\\{\}}
% Add ',fontsize=\small' for more characters per line
\usepackage{framed}
\definecolor{shadecolor}{RGB}{248,248,248}
\newenvironment{Shaded}{\begin{snugshade}}{\end{snugshade}}
\newcommand{\AlertTok}[1]{\textcolor[rgb]{0.94,0.16,0.16}{#1}}
\newcommand{\AnnotationTok}[1]{\textcolor[rgb]{0.56,0.35,0.01}{\textbf{\textit{#1}}}}
\newcommand{\AttributeTok}[1]{\textcolor[rgb]{0.77,0.63,0.00}{#1}}
\newcommand{\BaseNTok}[1]{\textcolor[rgb]{0.00,0.00,0.81}{#1}}
\newcommand{\BuiltInTok}[1]{#1}
\newcommand{\CharTok}[1]{\textcolor[rgb]{0.31,0.60,0.02}{#1}}
\newcommand{\CommentTok}[1]{\textcolor[rgb]{0.56,0.35,0.01}{\textit{#1}}}
\newcommand{\CommentVarTok}[1]{\textcolor[rgb]{0.56,0.35,0.01}{\textbf{\textit{#1}}}}
\newcommand{\ConstantTok}[1]{\textcolor[rgb]{0.00,0.00,0.00}{#1}}
\newcommand{\ControlFlowTok}[1]{\textcolor[rgb]{0.13,0.29,0.53}{\textbf{#1}}}
\newcommand{\DataTypeTok}[1]{\textcolor[rgb]{0.13,0.29,0.53}{#1}}
\newcommand{\DecValTok}[1]{\textcolor[rgb]{0.00,0.00,0.81}{#1}}
\newcommand{\DocumentationTok}[1]{\textcolor[rgb]{0.56,0.35,0.01}{\textbf{\textit{#1}}}}
\newcommand{\ErrorTok}[1]{\textcolor[rgb]{0.64,0.00,0.00}{\textbf{#1}}}
\newcommand{\ExtensionTok}[1]{#1}
\newcommand{\FloatTok}[1]{\textcolor[rgb]{0.00,0.00,0.81}{#1}}
\newcommand{\FunctionTok}[1]{\textcolor[rgb]{0.00,0.00,0.00}{#1}}
\newcommand{\ImportTok}[1]{#1}
\newcommand{\InformationTok}[1]{\textcolor[rgb]{0.56,0.35,0.01}{\textbf{\textit{#1}}}}
\newcommand{\KeywordTok}[1]{\textcolor[rgb]{0.13,0.29,0.53}{\textbf{#1}}}
\newcommand{\NormalTok}[1]{#1}
\newcommand{\OperatorTok}[1]{\textcolor[rgb]{0.81,0.36,0.00}{\textbf{#1}}}
\newcommand{\OtherTok}[1]{\textcolor[rgb]{0.56,0.35,0.01}{#1}}
\newcommand{\PreprocessorTok}[1]{\textcolor[rgb]{0.56,0.35,0.01}{\textit{#1}}}
\newcommand{\RegionMarkerTok}[1]{#1}
\newcommand{\SpecialCharTok}[1]{\textcolor[rgb]{0.00,0.00,0.00}{#1}}
\newcommand{\SpecialStringTok}[1]{\textcolor[rgb]{0.31,0.60,0.02}{#1}}
\newcommand{\StringTok}[1]{\textcolor[rgb]{0.31,0.60,0.02}{#1}}
\newcommand{\VariableTok}[1]{\textcolor[rgb]{0.00,0.00,0.00}{#1}}
\newcommand{\VerbatimStringTok}[1]{\textcolor[rgb]{0.31,0.60,0.02}{#1}}
\newcommand{\WarningTok}[1]{\textcolor[rgb]{0.56,0.35,0.01}{\textbf{\textit{#1}}}}
\usepackage{graphicx}
\makeatletter
\def\maxwidth{\ifdim\Gin@nat@width>\linewidth\linewidth\else\Gin@nat@width\fi}
\def\maxheight{\ifdim\Gin@nat@height>\textheight\textheight\else\Gin@nat@height\fi}
\makeatother
% Scale images if necessary, so that they will not overflow the page
% margins by default, and it is still possible to overwrite the defaults
% using explicit options in \includegraphics[width, height, ...]{}
\setkeys{Gin}{width=\maxwidth,height=\maxheight,keepaspectratio}
% Set default figure placement to htbp
\makeatletter
\def\fps@figure{htbp}
\makeatother
\setlength{\emergencystretch}{3em} % prevent overfull lines
\providecommand{\tightlist}{%
  \setlength{\itemsep}{0pt}\setlength{\parskip}{0pt}}
\setcounter{secnumdepth}{-\maxdimen} % remove section numbering
\ifluatex
  \usepackage{selnolig}  % disable illegal ligatures
\fi

\title{UCL policy project markdown}
\author{}
\date{\vspace{-2.5em}}

\begin{document}
\maketitle

\begin{Shaded}
\begin{Highlighting}[]
\FunctionTok{library}\NormalTok{(ggplot2)}
\NormalTok{summarized\_ecl\_data }\OtherTok{\textless{}{-}} \FunctionTok{read.csv}\NormalTok{(}\StringTok{"summarized\_ecl\_data.csv"}\NormalTok{)}

\NormalTok{h\_a\_goals\_comp }\OtherTok{\textless{}{-}} \FunctionTok{as.data.frame}\NormalTok{(}\FunctionTok{t}\NormalTok{(summarized\_ecl\_data[, }\FunctionTok{c}\NormalTok{(}\StringTok{"Total.Home.goals"}\NormalTok{, }\StringTok{"Total.Away.goals"}\NormalTok{)]))}
\NormalTok{h\_a\_goals\_comp}\SpecialCharTok{$}\NormalTok{goals }\OtherTok{\textless{}{-}} \FunctionTok{row.names}\NormalTok{(h\_a\_goals\_comp)}

\NormalTok{leg\_goals\_comp }\OtherTok{\textless{}{-}} \FunctionTok{as.data.frame}\NormalTok{(}\FunctionTok{t}\NormalTok{(summarized\_ecl\_data[, }\FunctionTok{c}\NormalTok{(}\StringTok{"Total.1st.leg.goals"}\NormalTok{, }\StringTok{"Total.2nd.leg.goals"}\NormalTok{)]))}
\NormalTok{leg\_goals\_comp}\SpecialCharTok{$}\NormalTok{goals }\OtherTok{\textless{}{-}} \FunctionTok{row.names}\NormalTok{(leg\_goals\_comp)}

\NormalTok{f\_leg\_home\_wins }\OtherTok{\textless{}{-}} \FunctionTok{as.data.frame}\NormalTok{(}\FunctionTok{t}\NormalTok{(summarized\_ecl\_data[, }\FunctionTok{c}\NormalTok{(}\StringTok{"X1st.leg.home.wins"}\NormalTok{, }\StringTok{"X1st.leg.home.loses"}\NormalTok{)]))}
\NormalTok{f\_leg\_home\_wins}\SpecialCharTok{$}\NormalTok{wins }\OtherTok{\textless{}{-}} \FunctionTok{row.names}\NormalTok{(f\_leg\_home\_wins)}

\NormalTok{s\_leg\_home\_wins }\OtherTok{\textless{}{-}} \FunctionTok{as.data.frame}\NormalTok{(}\FunctionTok{t}\NormalTok{(summarized\_ecl\_data[, }\FunctionTok{c}\NormalTok{(}\StringTok{"X2nd.leg.home.wins"}\NormalTok{, }\StringTok{"X2nd.leg.home.loses"}\NormalTok{)]))}
\NormalTok{s\_leg\_home\_wins}\SpecialCharTok{$}\NormalTok{wins }\OtherTok{\textless{}{-}} \FunctionTok{row.names}\NormalTok{(s\_leg\_home\_wins)}

\NormalTok{e\_p\_wins }\OtherTok{\textless{}{-}} \FunctionTok{as.data.frame}\NormalTok{(}\FunctionTok{t}\NormalTok{(summarized\_ecl\_data[, }\FunctionTok{c}\NormalTok{(}\StringTok{"Extratime.and.penalties.wins"}\NormalTok{, }\StringTok{"Extratime.and.penalties.loses"}\NormalTok{)]))}
\NormalTok{e\_p\_wins}\SpecialCharTok{$}\NormalTok{wins }\OtherTok{\textless{}{-}} \FunctionTok{row.names}\NormalTok{(e\_p\_wins)}

\NormalTok{a\_g\_wins }\OtherTok{\textless{}{-}} \FunctionTok{as.data.frame}\NormalTok{(}\FunctionTok{t}\NormalTok{(summarized\_ecl\_data[, }\FunctionTok{c}\NormalTok{(}\StringTok{"Away.goals.wins"}\NormalTok{, }\StringTok{"Away.goals.loses"}\NormalTok{)]))}
\NormalTok{a\_g\_wins}\SpecialCharTok{$}\NormalTok{wins }\OtherTok{\textless{}{-}} \FunctionTok{row.names}\NormalTok{(a\_g\_wins)}
\end{Highlighting}
\end{Shaded}

\hypertarget{motivation-and-problems-to-be-solved}{%
\subsection{Motivation and Problems to be
solved}\label{motivation-and-problems-to-be-solved}}

I am an avid football fan. I support Real Madrid and the most
prestegious competition in the UEFA Champions League for clubs.
Recently, they proposed an idea to change the away rule goal in extra
time for Knockout stage matches. This rule always intrigued me and I
want to know if its implementation had the desired outcome. The outcome
was supposed to be based on fairness where the home side had the
advantage because of home fans and familarity to the pitch, and to
eliminate that the away team goals are worth more when the scores are
level.

\hypertarget{hypothesis}{%
\subsection{Hypothesis}\label{hypothesis}}

The team that plays away last has the advantage since more goals tend to
be scored in this match and the away goals are worth more. This leads to
disproportionate amount of teams quantifying to the next round that
played the away game last.

\hypertarget{cleaning}{%
\subsection{Cleaning}\label{cleaning}}

\begin{enumerate}
\def\labelenumi{\arabic{enumi}.}
\tightlist
\item
  The files of the individual UCL seasons were name the same,
  ``champs.csv'', and stored in a separate folders defined by the year
  of that season. So to avoid conflicts when calling files, the csv file
  contain in each folder was renamed by combining it's original name
  with that of the folder it is contained in using python to automate
  the process.
\item
  Some files had extra columns that weren't relevant to analyst so they
  were removed and a new data frame created using python.
\item
  Each of these files that stored a single season was aggregated into
  one master sheet using python for analyst.
\item
  The master sheet was filter for specific information either using
  python or Microsoft Excel.
\item
  The data was scrubbed and any abnormal data was either verified and
  changed or removed.
\end{enumerate}

\hypertarget{results}{%
\subsection{Results}\label{results}}

From the graph below, we observe that the home team significantly
outscores the away team. The ratio of home goals to away goals was found
to be 7:4, which suggests the homes team has a substantial advantage.

\begin{Shaded}
\begin{Highlighting}[]
\FunctionTok{library}\NormalTok{(ggplot2)}
\FunctionTok{ggplot}\NormalTok{(}\AttributeTok{data=}\NormalTok{h\_a\_goals\_comp, }\FunctionTok{aes}\NormalTok{(}\AttributeTok{x=}\NormalTok{goals , }\AttributeTok{y=}\NormalTok{V1)) }\SpecialCharTok{+}
  \FunctionTok{geom\_bar}\NormalTok{(}\AttributeTok{stat=}\StringTok{"identity"}\NormalTok{) }\SpecialCharTok{+}
  \FunctionTok{xlab}\NormalTok{(}\StringTok{""}\NormalTok{) }\SpecialCharTok{+} \FunctionTok{ylab}\NormalTok{(}\StringTok{"No. of Goals"}\NormalTok{)}
\end{Highlighting}
\end{Shaded}

\includegraphics{UCL_project_markdown_files/figure-latex/unnamed-chunk-2-1.pdf}

As observed in the graph below, the amount of goals scored in the first
leg and the second leg were 5947 and 6361 respectively. As a percentage,
51.68\% of the goals were scored in the second leg. The p-value was
calculated to be 1.43\% and with a confidence level set at 95\%, the
null hypothesis that the amount of goals scored between the first and
second leg would be equal was rejected. The reason for this difference
could be the fact that in the second leg, the tie is decided and gives
sides the incentive to score.

\begin{Shaded}
\begin{Highlighting}[]
\FunctionTok{ggplot}\NormalTok{(}\AttributeTok{data=}\NormalTok{leg\_goals\_comp, }\FunctionTok{aes}\NormalTok{(}\AttributeTok{x=}\NormalTok{goals , }\AttributeTok{y=}\NormalTok{V1)) }\SpecialCharTok{+}
  \FunctionTok{geom\_bar}\NormalTok{(}\AttributeTok{stat=}\StringTok{"identity"}\NormalTok{) }\SpecialCharTok{+}
  \FunctionTok{xlab}\NormalTok{(}\StringTok{""}\NormalTok{) }\SpecialCharTok{+} \FunctionTok{ylab}\NormalTok{(}\StringTok{"No. of Goals"}\NormalTok{)}
\end{Highlighting}
\end{Shaded}

\includegraphics{UCL_project_markdown_files/figure-latex/unnamed-chunk-3-1.pdf}

The data also show teams that play away first won more games at home
than the team than played home first as shown in the graphs below. It
can be seen that they won more away games as well. This leads to the
argument that the home side advantage due to fan support plays a bigger
part in the second leg than the first leg generally by showing more
support.

\begin{Shaded}
\begin{Highlighting}[]
\FunctionTok{ggplot}\NormalTok{(}\AttributeTok{data=}\NormalTok{f\_leg\_home\_wins, }\FunctionTok{aes}\NormalTok{(}\AttributeTok{x=}\NormalTok{wins , }\AttributeTok{y=}\NormalTok{V1)) }\SpecialCharTok{+}
  \FunctionTok{geom\_bar}\NormalTok{(}\AttributeTok{stat=}\StringTok{"identity"}\NormalTok{) }\SpecialCharTok{+}
  \FunctionTok{xlab}\NormalTok{(}\StringTok{""}\NormalTok{) }\SpecialCharTok{+} \FunctionTok{ylab}\NormalTok{(}\StringTok{"No. of Wins"}\NormalTok{)}
\end{Highlighting}
\end{Shaded}

\includegraphics{UCL_project_markdown_files/figure-latex/unnamed-chunk-4-1.pdf}

\begin{Shaded}
\begin{Highlighting}[]
\FunctionTok{ggplot}\NormalTok{(}\AttributeTok{data=}\NormalTok{s\_leg\_home\_wins, }\FunctionTok{aes}\NormalTok{(}\AttributeTok{x=}\NormalTok{wins , }\AttributeTok{y=}\NormalTok{V1)) }\SpecialCharTok{+}
  \FunctionTok{geom\_bar}\NormalTok{(}\AttributeTok{stat=}\StringTok{"identity"}\NormalTok{) }\SpecialCharTok{+}
  \FunctionTok{xlab}\NormalTok{(}\StringTok{""}\NormalTok{) }\SpecialCharTok{+} \FunctionTok{ylab}\NormalTok{(}\StringTok{"No. of Wins"}\NormalTok{)}
\end{Highlighting}
\end{Shaded}

\includegraphics{UCL_project_markdown_files/figure-latex/unnamed-chunk-4-2.pdf}

The subset of results where extra-time and penalties were needed to
settle the tie, on 35 occasions the team that played home first won and
63 occasions the team that played away first won following the general
trend.

\begin{Shaded}
\begin{Highlighting}[]
\FunctionTok{ggplot}\NormalTok{(}\AttributeTok{data=}\NormalTok{e\_p\_wins, }\FunctionTok{aes}\NormalTok{(}\AttributeTok{x=}\NormalTok{wins , }\AttributeTok{y=}\NormalTok{V1)) }\SpecialCharTok{+}
  \FunctionTok{geom\_bar}\NormalTok{(}\AttributeTok{stat=}\StringTok{"identity"}\NormalTok{) }\SpecialCharTok{+}
  \FunctionTok{xlab}\NormalTok{(}\StringTok{""}\NormalTok{) }\SpecialCharTok{+} \FunctionTok{ylab}\NormalTok{(}\StringTok{"No. of Wins"}\NormalTok{)}
\end{Highlighting}
\end{Shaded}

\includegraphics{UCL_project_markdown_files/figure-latex/unnamed-chunk-5-1.pdf}

When the aggregate scores are even, and the away goal rule decides the
out come. On 80 occasions the team that played home first won and 78
occasions the team that played away first won. This was contrary to the
general trend where the teams that plays away first win more. In this
subset however, the team that played home first won or drew the first
match more than 90\% of the time. This means that a tie decided on away
goals favored conditions where the home team didn't lose and came into
the 2nd leg with an advantage or small disadvantage. When the factor in
that marginally more goals are scored in the second leg, the team that
played home first would likely score more away goals than the other side
since scores would be even at the end of the tie.

\begin{Shaded}
\begin{Highlighting}[]
\FunctionTok{ggplot}\NormalTok{(}\AttributeTok{data=}\NormalTok{a\_g\_wins, }\FunctionTok{aes}\NormalTok{(}\AttributeTok{x=}\NormalTok{wins , }\AttributeTok{y=}\NormalTok{V1)) }\SpecialCharTok{+}
  \FunctionTok{geom\_bar}\NormalTok{(}\AttributeTok{stat=}\StringTok{"identity"}\NormalTok{) }\SpecialCharTok{+}
  \FunctionTok{xlab}\NormalTok{(}\StringTok{""}\NormalTok{) }\SpecialCharTok{+} \FunctionTok{ylab}\NormalTok{(}\StringTok{"No. of Wins"}\NormalTok{)}
\end{Highlighting}
\end{Shaded}

\includegraphics{UCL_project_markdown_files/figure-latex/unnamed-chunk-6-1.pdf}

\hypertarget{conslusion-and-insights}{%
\subsection{Conslusion and Insights}\label{conslusion-and-insights}}

The three main advantages a team could have in this format is a tactical
advantage, away goal advantage and the home advantage. The tactical
advantage is what separates team from each other and should be the sole
merit of the team that allows them to win. This is why the away goal
rule was implemented to hypothetically eliminate this advantage.
however, it is shown through the data that the home advantage is
generally more significant due to the home side fans. Therefore, the
away goals rule has not been effect in it's implementation to reduce
this advantage. UEFA has removed this from the 2021/22 season going
forward. The problem with the advantage the fans give is that it's
dependent on other factors but the one that dominates the 2 leg tie is
the result of the 1st leg result. If their team loses, the can create
and atmosphere that's intimidating to the opposition and gives the home
team the edge. This can be seen in recent games such as PSG vs Barcelona
in 2017, Barcelona vs Roma in 2018, Barcelona vs Liverpool in 2019 or
Atletico Madrid vs Juventus in 2019. Since the tie is decided in the 2nd
leg, this gives the home crowd the advantage to influence the game even
more which, to me, is the reason why the team that plays home in the
second leg ends up winning the ties more often than not.

\hypertarget{solutions}{%
\subsection{Solutions}\label{solutions}}

As stated before, UEFA decided to eliminate the away goals rule which
may skew the results even more but the beauty of this action is that it
should gives the fans more incentive to show support. This empowers the
fans more than ever which is what professional football should be about.

\hypertarget{limitation-and-possible-errors}{%
\subsection{Limitation and Possible
Errors}\label{limitation-and-possible-errors}}

I don't know the procedure for choosing which team plays home first by
UEFA which may affect the results or be a reason to the outcomes
observed. This is also a high level analysts and much more insights
could be achieved with deep analysts into the data.

\end{document}
